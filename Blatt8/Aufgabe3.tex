\aufgabe 3

\begin{enumerate}
\item Die Lineare Transformation wird genutzt, um das nächste Segment der flowpipe zu berechnen.
\item Der Membershiptest wird genutzt, um die Einhaltung der geltenden Invarianten zu überprüfen.
\item Die Durchschnittsbildung und die Minkowski Summe werden für die Durchführung diskrete Sprünge verwendet. Die Durchschnittsbildung garantiert dabei die Einhaltung der relevanten Guards.
\item Das initiale Segment wird durch Bildung einer konvexen Hülle um die Vereinigung von X0 und der Menge der im ersten Zeitsegment erreichbaren Punkten gebildet.
\item Test for emptiness
\end{enumerate}
